\documentclass[12pt]{diazessay}

\usepackage{bashful}
\usepackage{hyperref}

%----------------------------------------------------------------------------------------
%	Comment this out if you do not have `texcount` installed on your $PATH
%----------------------------------------------------------------------------------------
%\bash
%command -v texcount &> /dev/null && texcount -sum -1 csci-724-paper.tex
%\END


%Shorthand formatting commands
\newcommand{\F}[1]{$\quad$\texttt{#1}}
\newcommand{\A}{$\alpha$}
\newcommand{\B}{$\beta$}
\newcommand{\Bool   }{\texttt{Bool}}
\newcommand{\Nat    }{\texttt{Natural}}
\newcommand{\Integer}{\texttt{Integer}}
\newcommand{\Double }{\texttt{Double}}
\newcommand{\List   }{\texttt{List}}
\newcommand{\Type   }{\texttt{Type}}

%----------------------------------------------------------------------------------------
%	TITLE SECTION
%----------------------------------------------------------------------------------------

\title{\texttt{\huge{Fuzzing NetHack for Fun and Prestige} \\ {\large A CSCI-795 Project Proposal}}} % Title and subtitle

\author{\texttt{{\Huge Team:}\\\vspace*{-0.5em} 
		Sabina Bhuiyan \\\vspace*{-0.5em} 
		Kyoungwoo Lee \\\vspace*{-0.5em}
		Chuanyao Lin \\\vspace*{-0.25em}
		Alex Washburn}} % Author and institution

\date{\texttt{\today}} % Date, use \date{} for no date

\pagestyle{empty}
%----------------------------------------------------------------------------------------

\begin{document}

\maketitle % Print the title section

\section*{Project Proposal}

For our project, we propose to perform application security testing of the program \emph{NetHack} \cite{NetHack_page} by using the fuzzing tool AFL \cite{AFL_page} to generate aberrant user inputs. 
\emph{NetHack} is an actively developed, open source terminal game forked from the game \emph{Hack} which itself was a clone of the game \emph{Rogue}. 
\emph{NetHack} can trace it's origins all the way back to it's initial release 1987, though arguably it could be considered as old as the game \emph{Hack} from which it was forked, released in 1982. 
\emph{NetHack} accepts keyboard inputs from the user to direct gameplay and updates the terminal to display the game state to the user.

American Fuzzy Lop (AFL) is an open source fuzzing tool.
Fuzzing is a venerable application security testing technique developed in the late '80s and published in 1990.
Fuzzing tools generate input which is directed towards the program and detect if the input was accepted or caused unexpected behavior.
Unexpected behavior can take many forms including, infinite loops, stack smashing, buffer overflows, arithmetic overflows or errors, and illegal program states.
Unexpected behavior that we are looking for include runtime exceptions, infinite loops, and illegal states.
Examples of illegal states we will look for include performing actions not allowed in the game even if the game has not crashed, such as like going through walls or spawning items.
Fuzzing tools can take many forms and modify program input in many ways.
Different fuzzing tools tend to specialize in a few related forms of input manipulation.
AFL relies on genetic algorithm techniques to mutate acceptable input into input triggering unexpected behavior.

The goal of this project is to use AFL to produce one or more \emph{minimal} sequences of ``player keyboard input'' which generate unexpected behavior in \emph{NetHack}.
Since \emph{NetHack} is open source, we intend to patch and submit pull requests for any defects we discover.
If the discovered defect(s) are notable, we will draft a manuscript and publish in an appropriate journal after the semester ends.

\clearpage


\section*{Motivation}

As gamers, we all have experiences of observing actions which are not designed to be possible in a game.
Whether this is wall hacks, save glitches, item duplication, speed-running skips, or just a means to gain an advantage in competitive online play, the experience becomes ubiquitous as playtime increases.
Notable examples are present in the game speed-running community.
The goal of speed running is to compelete a game as quickly as possible, and hence speed-runners are incentivised to find exploit which reduce playime of the game.
A common example is speed-runners using unconventional input sequences to bypass an in-game locked door, often skipping large sections of game dedicated to unlocking the door.
It is often surprising how these exploits are found.
Many profess manually play-testing the game until uncovering unexpected behavior that they find beneficial.
After taking the cybersecurity course and learning about the fuzzing software, we hope to see whether these tools might can be utilized as an automated method of finding out these game exploits.


\section*{Approach}

We will use AFL to fuzze the game \emph{NetHack} in search of unexpected behavior.
First we will generate a series of ``seed inputs'' for AFL which are accepted by \emph{NetHack} and generate expected behavior.
Then we will then allow AFL to mutate these seed inputs in to probe for unexpected behavior.
During the initial calibration of AFL we will manually monitor AFL's progress and adapt the seeding process as needed to ensure that AFL is effectively probing \emph{NetHack} for gameplay defects.

We will configure AFL to send textual inputs to \emph{NetHack} in a manner simulating player keyboard input.
\emph{NetHack} will interpret the keyboard input as gameplay commands and update it's internal state accordingly.
Given \emph{NetHack's} age, nearly 40 years old, we anticipate that finding unexpected behavior will not be an exercise in picking ``low hanging fruit,'' but rather will require some creativity on the part of our team in guiding AFL's fuzzing techniques with appropriate seeds.

Success of the project can easily be detected if AFL discovers an input sequence which causes \emph{NetHack} to loop indefinitely or to crash; via thrown runtime exception or OS intervention.
Additionally, we will check for AFL causing \emph{NetHack} to enter an ``unplayable'' state.
By this we mean that not input to \emph{NetHack} causes it's state to change.
We will test for entering an unplayable state by attempting to perform basic, automated, deterministic user inputs *after* an AFL fuzzing sequence has ended.
In any of the cases described above, we will record the triggering input sequence and attempt reduce it to a minimal triggering sequence.

Once one or more minimal triggering sequences have been discovered, we will begin a process of debugging \emph{NetHack} to locate the source of the defect.
This is possible due to \emph{NetHack} being an open source game.
The process of debugging the discovered unexpected behavior will give us insight into the nature of the defect.
We intend to, if possible within the time constraints of the semester, to patch and submit pull requests correcting the discovered defects.

The team expects to use their personal, commodity computer hardware to carry out this project.
As both AFL and \emph{NetHack} are time and space efficient programs, it is unlikely that team members will encounter hardware limitations which prevent progress of the project.
However, should hardware constraints be discovered or it is determined to be beneficial for the project, the team members will have access to the discrete arithmetic computing cluster at the American Museum of Natural History (AMNH).
This cluster has 256 Intel cores and 4TB of RAM allowing for work to be massively parallelized if the AFL fuzzing of \emph{NetHack} can be left in a most automated state.
This cluster at AMNH does not have good floating point arithmetic support, and hence is generally unused by researcher in the astronomy and biology fields.
Fortunately for the team, neither \emph{NetHack} nor AFL relies heavily on floating point arithmetic, making this cluster a useful and available resource.



\section*{Anticipated Results}
\label{results}

The expected outcome of the project is to uncover one or more defects in the game \emph{NetHack}, perform a root cause analysis to determine the source of each defect, and author a patch to correct each defect. 
A successful project will produce a catalogue of discovered defects. 
Each catalogue entry will contain the following: 

\begin{enumerate}

  \item Description of the triggering input sequence
  \item Description of the unexpected behavior
  \item Script to replicate the unexpected behavior
  \item Summarized code review of the source code involving the defect
  \item Classification of the defect
  \item File diff representing the patch correcting to the defect
  \item Reference to the pull request to apply the patch

\end{enumerate}

In the (unfortunate) event that no defects could be found in \emph{NetHack}, the team will author a report describing the techniques used and identifying the shortcomings of these techniques in the application security testing of \emph{NetHack}.


\section*{Division of Labor}

Each member will be using the fuzzing software to investigate for unexpected behavior in \emph{NetHack}.
As defects are discovered, one or more group members may be diverted from actively guiding AFL's fuzzing efforts to debugging \emph{NetHack} and performing root cause analysis of the game's source code to identify, classify, and correct the defect.
Where appropriate, ``ownership'' of a defect may be assigned to, or voluntarily assumed by, a team member who will then be responsible for completing the catalogue entry of the defect as described in the \nameref{results} section.
In general, the processes of authoring reports, completing literature review, preparting presentations, and presenting will be shared by all team members.


\section*{Timeline}

The project timeline is partially defined by the course syllabus where deadline are concerned and partially team defined where the syllabus does not give guidance.
The team has collaboratively discussed and drafted a tentative timeline of the project based on our assumptions of progress and integrated this with the deliverable deadlines laid out in the course syllabus. The current, tentative project timeline is as follows:

  \begin{enumerate}[label={}]
  	\item \texttt{2021-08-29:} Form team and brainstorm project ideas
  	\item \texttt{2021-09-05:} \textbf{Project Proposal due}
  	\item \texttt{2021-09-12:} Get AFL \& \emph{NetHack} working locally and interfaced
 	\item \texttt{2021-09-19:} Begin generating fuzzing inputs to \emph{NetHack}
  	\item \texttt{2021-09-26:} Start the project report draft
  	\item \texttt{2021-10-03:} Search for a cool demo breaking input for presentation
  	\item \texttt{2021-10-10:} \textbf{Midterm project presentations due}
  	\item \texttt{2021-10-17:} \textbf{Midterm project reports due}
  	\item \texttt{2021-10-24:} Apply parallelism input fizzing
  	\item \texttt{2021-10-31:} Root cause analysis of \emph{NetHack} source code
  	\item \texttt{2021-11-07:} Debugging \emph{NetHack}
  	\item \texttt{2021-11-14:} Author patch \& pull requests for discovered defects
  	\item \texttt{2021-11-21:} Start finalizing final draft
  	\item \texttt{2021-11-28:} Start project report final draft \& final presentation
  	\item \texttt{2021-12-05:} \textbf{Final project presentations due}
  	\item \texttt{2021-12-12:} \textbf{Final project reports due 5:35 pm}
  \end{enumerate}


\clearpage

\bibliographystyle{acm}
\bibliography{project-proposal}

\end{document}
