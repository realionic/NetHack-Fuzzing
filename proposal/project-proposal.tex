\documentclass[12pt]{diazessay}

\usepackage{bashful}
\usepackage{hyperref}

%----------------------------------------------------------------------------------------
%	Comment this out if you do not have `texcount` installed on your $PATH
%----------------------------------------------------------------------------------------
%\bash
%command -v texcount &> /dev/null && texcount -sum -1 csci-724-paper.tex
%\END


%Shorthand formatting commands
\newcommand{\F}[1]{$\quad$\texttt{#1}}
\newcommand{\A}{$\alpha$}
\newcommand{\B}{$\beta$}
\newcommand{\Bool   }{\texttt{Bool}}
\newcommand{\Nat    }{\texttt{Natural}}
\newcommand{\Integer}{\texttt{Integer}}
\newcommand{\Double }{\texttt{Double}}
\newcommand{\List   }{\texttt{List}}
\newcommand{\Type   }{\texttt{Type}}

%----------------------------------------------------------------------------------------
%	TITLE SECTION
%----------------------------------------------------------------------------------------

\title{\texttt{\huge{Fuzzing NetHack for Fun and Prestige} \\ {\large A CSCI-795 Project Proposal}}} % Title and subtitle

\author{\texttt{{\Huge Team:}\\\vspace*{-0.5em} 
		Sabina Bhuiyan \\\vspace*{-0.5em} 
		Kyoungwoo Lee \\\vspace*{-0.5em}
		Chuanyao Lin \\\vspace*{-0.25em}
		Alex Washburn}} % Author and institution

\date{\texttt{\today}} % Date, use \date{} for no date

\pagestyle{empty}
%----------------------------------------------------------------------------------------

\begin{document}

\maketitle % Print the title section

\section*{Project Proposal}

For our project, we propose to perform application security testing of the program \emph{NetHack} by using the fuzzing tool AFL to generate aberrant user inputs. 
\emph{NetHack} is an actively developed, open source terminal game forked from the game \emph{Hack} which itself was a clone of the game \emph{Rogue}. 
\emph{NetHack} can trace it's origins all the way back to it's initial release 1987, though arguably it could be considered as old as the game \emph{Hack} from which it was forked, released in 1982. 
\emph{NetHack} accepts keyboard inputs from the user to direct gameplay and updates the terminal to display the game state to the user.

American Fuzzy Lop (AFL) is an open source fuzzing tool.
Fuzzing is a venerable application security testing technique developed in the late '80s and published in 1990.
Fuzzing tools generate input which is directed towards the program and detect if the input was accepted or caused unexpected behavior.
Unexpected behavior can take many forms including, infinite loops, stack smashing, buffer overflows, arithmetic overflows or errors, and illegal program states.
Unexpected behavior that we are looking for include runtime exceptions, infinite loops, and illegal states.
Examples of illegal states we will look for include performing actions not allowed in the game even if the game has not crashed, such as like going through walls or spawning items.
Fuzzing tools can take many forms and modify program input in many ways.
Different fuzzing tools tend to specialize in a few related forms of input manipulation.
AFL relies on genetic algorithm techniques to mutate acceptable input into input triggering unexpected behavior.

The goal of this project is to use AFL to produce one or more \emph{minimal} sequences of ``player keyboard input'' which generate unexpected behavior in \emph{NetHack}.
Since \emph{NetHack} is open source, we intend to patch and submit pull requests for any defects we discover.
If the discovered defect(s) are notable, we will draft a manuscript and publish in an appropriate journal after the semester ends.

\clearpage


\section*{Motivation}

As gamers, we might have experiences of committing some actions that were not allowed in the game.
For example, gamers use some commands to break into the locked door in the game.
In fact, we are often surprised how these bugs are found, and end up guessing whether these bugs are revealed from the game design team.
After taking the cybersecurity course and knowing about the fuzzing software, we hope to see whether these tools might be one of the methods to find out those bugs.


\section*{Approach}

\begin{enumerate}
  \item Use AFL to generate aberrant inputs to NetHack, simulating textual user gameplay inputs.
  \item Author a series of seed inputs for AFL to mutate.
  \item Record and case in which the game crashes
  \item Record any cases in which the game hangs
  \item Record any cases in which the game enters an “unplayable state”
  \item “Unplayable” will be tested by attempting to perform basic, automated, deterministic user inputs *after* an AFL fuzzing sequence has ended.
  \item Anticipated resources are the student’s commodity laptops, as both AFL and NetHack are time and memory efficient programs.
  \item However, the students have access to the 256 Intel core, 4TB RAM, discrete arithmetic computing cluster at AMNH, which could be used to massively parallelize work once AFL fuzzing has been fully operationalized.
\end{enumerate}


\section*{Expected Outcomes/Deliverables}

\begin{enumerate}
  \item Desired outcome is one or more detected user input vulnerabilities.
  \item Stack smashing, buffer overflows, infinite loops, etc.
  \item Author a patch to fix the user input vulnerabilities.
  \item Submit a pull request to the open source project of NetHack.
\end{enumerate}


\section*{Division of Labor}


\section*{Timeline}

  \begin{enumerate}[label={}]
  	\item \texttt{2021-08-29:} Form team and brainstorm project ideas
  	\item \texttt{2021-09-05:} \textbf{Project Proposal due}
  	\item \texttt{2021-09-12:} Get AFL \& NetHack working locally and interfaced
 	\item \texttt{2021-09-19:} Begin generating fuzzing inputs to NetHack
  	\item \texttt{2021-09-26:} Start the project report draft
  	\item \texttt{2021-10-03:} Search for a cool demo breaking input for presentation
  	\item \texttt{2021-10-10:} \textbf{Midterm project presentations due}
  	\item \texttt{2021-10-17:} \textbf{Midterm project reports due}
  	\item \texttt{2021-10-24:} Apply parallelism input fizzing
  	\item \texttt{2021-10-31:} Inspect NetHack source code
  	\item \texttt{2021-11-07:} Debugging NetHack
  	\item \texttt{2021-11-14:} Author patch \& pull requests for discovered defects
  	\item \texttt{2021-11-21:} Start finalizing final draft
  	\item \texttt{2021-11-28:} Start project report final draft \& final presentation
  	\item \texttt{2021-12-05:} \textbf{Final project presentations due}
  	\item \texttt{2021-12-12:} \textbf{Final project reports due 5:35 pm}
  \end{enumerate}


\end{document}
